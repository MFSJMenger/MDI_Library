\hypertarget{index_overview_sec}{}\section{Overview}\label{index_overview_sec}
The Mol\-S\-S\-I Driver Interface (M\-D\-I) library enables codes to interoperate via the M\-D\-I.\hypertarget{index_source_sec}{}\section{Source Code}\label{index_source_sec}
The source code of the M\-D\-I library is available at Git\-Hub at \href{https://github.com/MolSSI/molssi_driver_interface}{\tt https\-://github.\-com/\-Mol\-S\-S\-I/molssi\-\_\-driver\-\_\-interface}\hypertarget{index_commands_sec}{}\section{Commands}\label{index_commands_sec}
The following is a list of commands that are officially part of the M\-D\-I standard.\hypertarget{index_send_name}{}\subsection{$<$\-N\-A\-M\-E}\label{index_send_name}
The engine sends the driver a string of length {\ttfamily M\-D\-I\-\_\-\-N\-A\-M\-E\-\_\-\-L\-E\-N\-G\-T\-H} that corresponds to the argument of {\ttfamily -\/name} in the M\-D\-I initialization options. This argument allows a driver to identify the purpose of connected engine codes within the simulation. For example, a particular Q\-M/\-M\-M driver might require a connection with a single M\-M code and a single Q\-M code, with the expected name of the M\-M code being \char`\"{}\-M\-M\char`\"{} and the expected name of the Q\-M code being \char`\"{}\-Q\-M\char`\"{}. After initializing M\-D\-I and accepting communicators to the engines, the driver can use this command to identify which of the engines is the M\-M code and which is the Q\-M code.\hypertarget{index_md_init}{}\subsection{M\-D\-\_\-\-I\-N\-I\-T}\label{index_md_init}
The engine performs any initialization operations that are necessary before an M\-D simulation can be time propagated. These initialization operations {\bfseries  may change the engine's nuclear coordinates } under certain circumstances, such as if the S\-H\-A\-K\-E algorithm is used.\hypertarget{index_recv_natoms}{}\subsection{$<$\-N\-A\-T\-O\-M\-S}\label{index_recv_natoms}
The engine sends the driver a single integer corresponding to the number of nuclear centers in the engine's system.\hypertarget{index_recv_types}{}\subsection{$<$\-N\-T\-Y\-P\-E\-S}\label{index_recv_types}
The engine sends the driver a single integer corresponding to the number of different types of nuclear centers (e.\-g. \char`\"{}\-H\char`\"{}, \char`\"{}\-He\char`\"{}, \char`\"{}\-C\char`\"{}, \char`\"{}\-O\char`\"{}, etc.) in the engine's system.\hypertarget{index_recv_masses}{}\subsection{$<$\-M\-A\-S\-S\-E\-S}\label{index_recv_masses}
The engine sends the driver a number of double precision floats that correspond to the mass of each of the nuclear center types. The number of double precision floats sent is equal to the value of the integer sent in response to the $<$N\-T\-Y\-P\-E\-S command. The values are provided in sequentially ascending order of type index (see the {\ttfamily $<$T\-Y\-P\-E\-S} command).\hypertarget{index_set_cell}{}\subsection{$>$\-C\-E\-L\-L}\label{index_set_cell}
The driver sends a set of cell vectors to the engine, which immediately resizes its simulation cell in response. In the case of a quantum chemistry code that uses a plane wave basis set, the engine will either immediately recalculate the g-\/vectors or schedule the g-\/vectors to be recalculated at the beginning of the next S\-C\-F command.\hypertarget{index_recv_cell}{}\subsection{$<$\-C\-E\-L\-L}\label{index_recv_cell}
The engine sends a set of cell vectors, in the same format as specified for the {\ttfamily $>$C\-E\-L\-L} command. 